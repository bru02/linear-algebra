\input la4dm.tex
\title{Diagonal matrices}
\subtitle{The eignevalues of a matrix}
\date{December 2, 2024}
\begin{document}

\frame{\titlepage}
%\section*{Outline:}
\frame{\frametitle{Outline}
\tableofcontents}

\frame{\frametitle{Important}
Taking part on the written exam at \alert{10.00 am. of December 19, 2024} is compulsory!
}
\section{Notion of Eigenvalue and Eigenvector}
\subsection{Definition}
\frame{\frametitle{Eigenvalue and Eigenvector}
\begin{definition}
    The scalar $\lambda \in \mathbb{R}$ is called an \alert{eigenvalue} of the matrix 
    $\mathbf{A}\in \R^{n\times n}$, 
    if there exists a nonzero vector $\mathbf{s}\in\R^{n}$ for which
    \begin{equation*}
    \mathbf{As}=\lambda \mathbf{s}
    \end{equation*}
    holds. 
    The vector $\mathbf{s}$ is said to be the \alert{eigenvector} corresponding to the eigenvalue $\lambda $
    of matrix $\mathbf{A}.$ 
\end{definition}
}

\frame{\frametitle{The set of eigenvectors form a subspace}
If both $\mathbf{s}$ and $\mathbf{t}$ are eigenvectors corresponding to the eigenvalue $\lambda $ then their any nontrivial linear combination $\alpha \mathbf{s}+\beta \mathbf{t}$ is also an eigenvector with the same eigenvalue  $\lambda $:
\begin{equation*}
    \mathbf{A}\left( \alpha \mathbf{s}+\beta \mathbf{t}\right) =
    \alpha \mathbf{As}+\beta  \mathbf{At}=\alpha \lambda \mathbf{s}+\beta \lambda \mathbf{t}=
    \lambda \left( \alpha \mathbf{s}+\beta \mathbf{t}\right) .
\end{equation*}
}

\begin{frame}
    \frametitle{Dependendent system}
    The eigenvectors corresponding to the eigenvalue $\lambda $ are the nonzero solutions of the homogeneous system of linear equations 
$$\left( \mathbf{A}-\lambda \mathbf{E}\right) \mathbf{x}=\mathbf{0}.$$

A homogeneous system of linear equations with a square coefficient matrix has nonzero solution 
if and only if its columns form a dependent system.

It follows that $\lambda$ is an eigenvalue of $\mathbf{A}$ if and only if 
the columns of matrix
\[
\mathbf{A}-\lambda\mathbf{E}
\]
is a linearly dependent system.
\end{frame}



\section{Determine the eigenvalues and the eigenvectors}

\begin{frame}
\begin{example} Find the eigenvalues and the corresponding eigenvectors of the following matrix:
\begin{equation*}
\mathbf{A}=\left[\begin{array}{rrr}
1 & 0 & -1 \\
0 & 1 & 1 \\
2 & 0 & -2
\end{array}\right]
\end{equation*}
\end{example}
We find the values of the parameter $t$ for which the matrix $\mathbf{A}-t\mathbf{E}$ is singular and with the obtained values we find the nonzero solutions of the homogeneous system of linear equations $(\mathbf{A}-t\mathbf{E})\mathbf{x}=\mathbf{0}.$
\end{frame}

\subsection{Basis transformation}
\frame{\frametitle{Find the eigenvalues and the eigenvector with basis transformation}
Using elementary basis transformation we solve the homogeneous system of linear equations $(\mathbf{A}-t\mathbf{E})\mathbf{x}=\mathbf{0}$ for each value of the parameter $t$ for which the coefficient matrix is singular:
$$ 
\begin{array}{c|ccc}
&x_1&x_2&x_3 \\ \hline
&1-t&0&\fbox{$-1$}\\
&0&1-t&1\\
&2&0&-2-t
\end{array}\rightarrow \begin{array}{c|cc}
&x_1&x_2 \\ \hline
x_3&t-1&0\\
&1-t&1-t\\
&t^2+t&0
\end{array}$$
}

\frame{\frametitle{}
If $t=1$ then we can continue as follows:
$$\begin{array}{c|cc}
&x_1&x_2 \\ \hline
x_3&0&0\\
&0&0\\
&\fbox{$2$}&0
\end{array}\rightarrow \begin{array}{c|cc}
&x_2 \\ \hline
x_3&0\\
&0\\
x_1&0
\end{array}$$
From this table we obtain that at $t=1$ the coefficient matrix is singular and the solution of the homogeneous system of linear equations $(\mathbf{A}-\mathbf{E})\mathbf{x}=\mathbf{0}$
$$\left[\begin{array}{c}
x_1\\
x_2\\
x_3
\end{array}\right]=\left[\begin{array}{c}
0\\
1\\
0
\end{array}\right]\tau\quad \tau\in \R\ ,$$
that is $1$ is an eigenvalue and the above vectors are the corresponding eigenvectors for all $\tau \neq 0.$
}
\frame{\frametitle{}
If $t\neq 1$ then 
$$\begin{array}{c|cc}
&x_1&x_2 \\ \hline
x_3&t-1&0\\
&1-t&\fbox{$1-t$}\\
&t^2+t&0
\end{array}\rightarrow \begin{array}{c|cc}
&x_1 \\ \hline
x_3&t-1\\
x_2&1\\
&t^2+t
\end{array}$$
and we obtain that the coefficient matrix is also singular if $t=0$ or $t=-1.$

One can see, from the table, that the eigenvectors corresponding to the eigenvalue $0$
are 
$$\left[\begin{array}{c}
x_1\\
x_2\\
x_3
\end{array}\right]=\left[\begin{array}{c}
1\\
-1\\
1
\end{array}\right]\tau\quad \tau\in \R\ \tau\neq 0$$
and the eigenvectors corresponding to the eigenvalue $-1$ are 
$$\left[\begin{array}{c}
x_1\\
x_2\\
x_3
\end{array}\right]=\left[\begin{array}{c}
1\\
-1\\
2
\end{array}\right]\tau\quad \tau\in \R\ \tau\neq 0.$$

}
\frame{\frametitle{Alert}
There are real matrices which has no real eigenvalue, for instance the matrix
\begin{alertblock}{Matrix having no real eigenvalue}
\begin{equation*}
\mathbf{A}=\left[\begin{array}{rr}
0 & 1 \\
-1 & 0
\end{array}\right]
\end{equation*}
has no real eigenvalue,
because the columns of matrix $A-tE$ always form an independent system.
\end{alertblock}}

\section{Characteristic polynomial}

\frame{
    \begin{definition}
        If $A$ is a square matrix then the polynomial
        \[
            k\left( t \right)=
            \det\left( A-tI \right)
        \]
        is called the \emph{characteristic polynomial} of the matrix $A$.
    \end{definition}
\begin{theorem}
    Let $A$ be a square matrix and consider a real number $\lambda$.
    The number $\lambda$ is an eigenvalue of matrix $A$ if and only if $k\left( \lambda \right)=0$.
\end{theorem}
Verify that similar matrices have the same characteristic polynomial.
}
\frame{\frametitle{Two alternatives looking for eigenvalues and eigenvectors}

Now there are two ways to find the eigenvalues and the corresponding eigenvectors:
\begin{description}[<+->]
\item[First]
We find the values of the parameter $t$ for which the matrix $\mathbf{A}-t\mathbf{E}$ is singular and with the obtained values we find the nonzero solutions of the homogeneous system of linear equations $(\mathbf{A}-t\mathbf{E})\mathbf{x}=\mathbf{0},$ or
\item[Second]
We determine the roots of the characteristic polynomial of $\mathbf{A},$ those are the eigenvalues of $\mathbf{A}$ and then solving the homogeneous system of linear equations
$\left( \mathbf{A}-\lambda \mathbf{E}\right) \mathbf{x}=\mathbf{0}$ for each root we obtain the corresponding eigenvectors.
\end{description}
}
\frame{\frametitle{2nd method to find the eigenvalues of a matrix}
The characteristic equation:
\begin{equation*}
\left| \mathbf{A}-t \mathbf{E}\right| =\left|
\begin{array}{ccc}
1-t & 0 & -1 \\
0 & 1-t & 1 \\
2 & 0 & -2-t
\end{array}
\right| =-t ^{3}+t =0
\end{equation*}
Its solutions: $t_{1}=0$, $t_{2}=-1$ and $t_{3}=1$ are the eigenvalues of the matrix $\mathbf{A}.$
}

\frame{\frametitle{}
The eigenvectors belonging to the eigenvalue $0$ are the nonzero solutions of the homogeneous system of linear equations $\left( A-0\cdot \mathbf{E}\right)
\mathbf{x}=\mathbf{0}.$ Thus the system to be solved:
\begin{equation*}
\left[\begin{array}{rrr}
1 & 0 & -1 \\
0 & 1 & 1 \\
2 & 0 & -2
\end{array}\right] \left[\begin{array}{c}
x_{1} \\
x_{2} \\
x_{3}
\end{array}\right] =\left[\begin{array}{c}
0 \\
0 \\
0
\end{array}\right] .
\end{equation*}
The solutions: $x_{1}=\tau,$ $x_{2}=-\tau,$ $x_{3}=\tau $\quad
$\left( \tau \in \R\right) .$ Therefore the eigenvectors belonging to the eigenvalue $t_{1}=0$ are
\begin{equation*}
\left[\begin{array}{r}
1 \\
-1 \\
1
\end{array}\right] \tau,\quad \left( \tau\in \R\setminus \{ 0\}\right)
\end{equation*}
}
\frame{\frametitle{}
The eigenvectors belonging to the eigenvalue $-1$ are the nonzero solutions of the homogeneous system of linear equations $\left( A+1\cdot \mathbf{E}\right)
\mathbf{x}=\mathbf{0}.$ Thus the system to be solved:
\begin{equation*}
\left[\begin{array}{rrr}
2 & 0 & -1 \\
0 & 2 & 1 \\
2 & 0 & -1
\end{array}
\right] \left[\begin{array}{c}
x_{1} \\
x_{2} \\
x_{3}
\end{array}\right] =\left[\begin{array}{c}
0 \\
0 \\
0
\end{array}\right].
\end{equation*}
The solutions: $x_{1}=\tau,$ $x_{2}=-\tau,$ $x_{3}=2\tau$\quad
$\left( \tau\in \R\right).$ Therefore the eigenvectors belonging to the eigenvalue $t_{2}=-1$:
\begin{equation*}
\left[\begin{array}{r}
1 \\
-1 \\
2
\end{array}\right] \tau,\quad \left( \tau\in \R\setminus \{ 0\}\right) .
\end{equation*}
}

\frame{\frametitle{}
The eigenvectors belonging to the eigenvalue $1$ are the nonzero solutions of the homogeneous system of linear equations $\left( A-1\cdot \mathbf{E}\right)
\mathbf{x}=\mathbf{0},$ that is,
\begin{equation*}
\left[\begin{array}{rrr}
0 & 0 & -1 \\
0 & 0 & 1 \\
2 & 0 & -3
\end{array}\right] \left[\begin{array}{c}
x_{1} \\
x_{2} \\
x_{3}
\end{array}
\right] =\left[\begin{array}{c}
0 \\
0 \\
0
\end{array}\right] .
\end{equation*}
Solving the system the eigenvectors :
\begin{equation*}
\left[\begin{array}{c}
0 \\
1 \\
0
\end{array}\right] \tau,\quad \left( \tau\in \R\setminus \{ 0\}\right) .
\end{equation*} 
}

\frame{\frametitle{Alert}
There are real matrices which has no real eigenvalue, for instance the matrix
\begin{alertblock}{Matrix having no real eigenvalue}
\begin{equation*}
\mathbf{A}=\left[\begin{array}{rr}
0 & 1 \\
-1 & 0
\end{array}\right]
\end{equation*}
has the characteristic polynomial $t^2+1,$ therefore it has no real root.
\end{alertblock}
}
\section{Diagonal matrix}

\frame{\frametitle{Diagonal matrix}
\begin{definition}
A diagonal matrix is a square matrix, whose entries outside of the main diagonal are zeros. They will be denoted by
\begin{equation*}
\mathbf{D}=\text{diag}\left< d_{1},\ldots ,d_{n}\right>,
\end{equation*}
$d_{1},\ldots ,d_{n}$ are the entries in the main diagonal. 
\end{definition}
The eigenvalues of a diagonal matrix are just the entries in the main diagonal.
}

\frame{\frametitle{Diagonalizable matrix}
\begin{definition}
A square matrix $\mathbf{A}$ of order $n$ is said to be diagonalizable if there exists a regular matrix  $\mathbf{P}$ and a diagonal matrix $\mathbf{D}$ such that
\begin{equation*}
\mathbf{P}^{-1}\mathbf{AP}=\mathbf{D}=\text{diag}\left< d_{1},\ldots ,d_{n}\right> .
\end{equation*}
\end{definition}
For example the matrix 
\begin{equation*}
\mathbf{A}=\left(
\begin{array}{cc}
3 & 1 \\
2 & 4
\end{array}
\right)
\end{equation*}
is diagonalizable, because
\begin{eqnarray*}
\mathbf{P}^{-1}\mathbf{AP} =-\dfrac{1}{3}\left[
\begin{array}{rr}
2 & -1 \\
-1 & -1
\end{array}\right] \left[\begin{array}{cc}
3 & 1 \\
2 & 4
\end{array}\right] \left[\begin{array}{rr}
-1 & 1 \\
1 & 2
\end{array}\right] 
=\left[\begin{array}{cc}
2 & 0 \\
0 & 5
\end{array}\right]
\end{eqnarray*}
\pause
It is fascinating. \alert{:)} How can we find the matrix $\mathbf{P}$, above?
}

\frame{\frametitle{A necessary and sufficient condition}
\begin{theorem} \label{diag}
A square matrix $\mathbf{A}$ of order $n$ is diagonalizable if and only if it has $n$ linearly independent eigenvectors $\mathbf{x}_{1},\ldots ,\mathbf{x}_{n}.$ In this case
\begin{equation*}
\mathbf{P}^{-1}\mathbf{AP}=\left[\begin{array}{cccc}
\lambda _{1} & 0 & \ldots & 0 \\
0 & \lambda _{2} & \ldots & 0 \\
\vdots & \vdots &  & \vdots \\
0 & 0 & \ldots & \lambda _{n}
\end{array}\right],
\end{equation*}
where the columns of $\mathbf{P}$ are $\mathbf{x}_{1},\ldots ,\mathbf{x}_{n},$ and 
$\lambda _{1},\ldots ,\lambda _{n}$ scalars are the corresponding eigenvalues. 
\end{theorem}
}

\frame{\frametitle{Proof}
If $\mathbf{x}_{1},\ldots ,\mathbf{x}_{n}$ are linearly independent eigenvectors of
$\mathbf{A}$ and $\lambda _{1},\ldots ,\lambda _{n}$ are the corresponding eigenvalues, that is,
\[
\mathbf{Ax}_{1} =\lambda_{1}\mathbf{x}_{1},\quad
\mathbf{Ax}_{2} =\lambda_{2}\mathbf{x}_{2},\quad
\dots,\quad
\mathbf{Ax}_{n} =\lambda_{n}\mathbf{x}_{n}.
\]
then let $\mathbf{P=}\left[ \mathbf{x}_{1},\ldots,\mathbf{x}_{n}\right].$ 
Thus
\begin{multline*}
\mathbf{AP} =
\mathbf{A}\left[ \mathbf{x}_{1},\ldots
,\mathbf{x}_{n}\right]
=
\left[ \mathbf{Ax}_{1},\ldots ,\mathbf{Ax}_{n}\right]
=
\left[ \lambda _{1}\mathbf{x}_{1},\ldots ,\lambda _{n}\mathbf{x}_{n}\right]
=\\
\left[ \mathbf{x}_{1},\ldots ,\mathbf{x}_{n}\right] \text{diag}\left<\lambda _{1},\ldots ,\lambda _{n}\right>
=
\mathbf{P}\text{diag}\left< \lambda _{1},\ldots ,\lambda_{n}\right>,
\end{multline*}
therefore multiplying both sides by the inverse of $\mathbf{P}$, we obtain the identity required:
\begin{equation*}
\mathbf{P}^{-1}\mathbf{AP}=\text{diag}\left< \lambda _{1},\ldots ,\lambda _{n}\right> .
\end{equation*}
}

\frame{\frametitle{Proof}
Conversely, if 
\begin{equation*}
\mathbf{P}^{-1}\mathbf{AP}=\text{diag}\left< \lambda _{1},\ldots ,\lambda _{n}\right> ,
\end{equation*}
then
$$\mathbf{AP}=\mathbf{P}\text{diag}\left< \lambda _{1},\ldots ,\lambda _{n}\right>,$$
that is, for each column $\mathbf{x}_i$ $(i=1,\ldots ,n)$ of $\mathbf{P}$ the equation
$\mathbf{Ax}_i=\lambda_i\mathbf{x}_i$ holds, verifying that $\lambda_i$ is an eigenvalue of
$\mathbf{A}$ and a corresponding eigenvector is $\mathbf{x}_i .$ 
}

\frame{\frametitle{}
\begin{example}
Decide if the matrix $
%\begin{equation*}
\mathbf{A}=\left(
\begin{array}{cc}
3 & 1 \\
2 & 4
\end{array}
\right)$
%\end{equation*}
is diagonalizable! If it is diagonalizable then give an invertible matrix $\mathbf{P}$ for which $\mathbf{P}^{-1}\mathbf{AP}$ is a diagonal matrix.
\end{example}
\pause
The eigenvalues of $\mathbf{A}$ are $\lambda_{1}=2$  and $\,\lambda _{2}=5,$ furthermore the corresponding linearly independent eigenvectors:
\begin{equation*}
\mathbf{v}_{1}=\left[\begin{array}{r}
-1 \\
1
\end{array}\right] \quad \mbox{and}\quad \mathbf{v}_{2}=\left[\begin{array}{c}
1 \\
2
\end{array}\right]
\end{equation*}\pause
Hence by the previous theorem $\mathbf{A}$ is a diagonalizable matrix.

If $\mathbf{P}=\left[\begin{array}{rr}
-1 & 1 \\
1 & 2
\end{array}\right], $ then
\begin{eqnarray*}
\mathbf{P}^{-1}\mathbf{AP} =-\dfrac{1}{3}\left[
\begin{array}{rr}
2 & -1 \\
-1 & -1
\end{array}\right] \left[\begin{array}{cc}
3 & 1 \\
2 & 4
\end{array}\right] \left[\begin{array}{rr}
-1 & 1 \\
1 & 2
\end{array}\right]  
=\left[\begin{array}{cc}
2 & 0 \\
0 & 5
\end{array}\right]
\end{eqnarray*}

}

\frame{\frametitle{An important special case}
The necessary and sufficient condition for a matrix is being diagonalizable in theorem (\ref{diag}) can not be easily checked. Therefore it is an important result -- can be proved using mathematical induction -- that eigenvectors of a square matrix $\mathbf{A}$ corresponding to different eigenvalues are linearly independent, from which it immediately follows that an 
$n\times n$ real matrix $\mathbf{A}$ having $n$ different real eigenvalues is diagonalizable.
Of course the above condition is not necessary one.
}
\frame{\frametitle{Example}
Consider the matrix:
\begin{equation*}
\mathbf{A}=\left[\begin{array}{ccc}
2 & 1 & 0 \\
0 & 1 & 0 \\
0 & 1 & 2
\end{array}\right]
\end{equation*}
This matrix has only two different eigenvalues $\lambda _{1}=2$ and $\lambda_{2}=1$.
%(where  $\lambda _{1}=2$ is a root of the characteristic polynomial with multiplicity two).

The linearly independent eigenvectors corresponding to the eigenvalue $\lambda _{1}=2$ are: $\mathbf{v}_{1}=\left[\begin{array}{c}
1 \\
0 \\
0
\end{array}\right]$\quad\mbox{and}\quad $\mathbf{v}_{2}=\left[\begin{array}{c}
0 \\
0 \\
1
\end{array}\right], $
while the eigenvector belonging to the eigenvalue $\lambda _{2}=1$ is 
$\mathbf{v}_{3}=\left[\begin{array}{r}
1 \\
-1 \\
1
\end{array}\right] .$
Thus the matrix has three linearly independent eigenvectors, therefore it is diagonalizable.
}

\section{Symmetric matrices}
\frame{\frametitle{Symmetric matrices}
The square matrices occurring in economical applications are often symmetric, therefore it seems to be reasonable to study the question if the symmetric matrices are diagonalizable.

Recall that a real matrix $\mathbf{A}$ is said to be symmetric if $\mathbf{A}^\top=\mathbf{A}$ and it is  orthogonal if $\mathbf{A}^\top=\mathbf{A}^{-1}$.


\begin{theorem}

\begin{description}
\item[-]   The characteristic equation of an $n\times n$ symmetric matrix has $n$ real roots counting their multiplicity as well.

\item[-]   The eigenvectors corresponding to different eigenvalues of a symmetric matrix are orthogonal.
\end{description}
\end{theorem}
}

\frame{\frametitle{The spectral theorem of symmetric matrices}
\begin{theorem}
The symmetric matrices are diagonalizable, furthermore to each symmetric matrix $\mathbf{A}$ there exists an orthogonal matrix $\mathbf{U},$ such that
\begin{equation*}
\mathbf{U}^{-1}\mathbf{AU}=\mbox{diag}\left< \lambda _{1},\ldots ,\lambda_{n}\right> ,
\end{equation*}
where $\lambda _{1},\ldots ,\lambda _{n}$ are the eigenvalues of $\mathbf{A}.$
\end{theorem}
\begin{example}
Find the orthogonal matrix  $\mathbf{U}$ to the symmetric matrix $\mathbf{A}$ for which the matrix $\mathbf{U}^\top\mathbf{AU}$ is a diagonal one.
\begin{equation*}
\mathbf{A}=\left[\begin{array}{ccc}
3 & 0 & 0 \\
0 & 0 & 1 \\
0 & 1 & 0
\end{array}\right]
\end{equation*}
\end{example}
}

\frame{\frametitle{}
The eigenvalues of the matrix are $\lambda _{1}=-1,\,\lambda _{2}=1$
and $\lambda _{3}=3, $ and the corresponding eigenvectors are
\begin{equation*}
\mathbf{v}_{1}=\left[\begin{array}{c}
0 \\
-1 \\
1
\end{array}\right] \,,\mathbf{v}_{2}=\left[\begin{array}{c}
0 \\
1 \\
1
\end{array}\right]\quad \mbox{and}\quad\mathbf{v}_{3}=\left[\begin{array}{c}
1 \\
0 \\
0
\end{array}\right] .
\end{equation*}
The normed eigenvectors are the columns of the orthogonal matrix $\mathbf{U}.$ Thus 
\begin{equation*}
\mathbf{U}=\left[
\begin{array}{ccc}
0 & 0 & 1 \\
-\frac{1}{\sqrt{2}} & \frac{1}{\sqrt{2}} & 0 \\
\frac{1}{\sqrt{2}} & \frac{1}{\sqrt{2}} & 0
\end{array}\right]
\end{equation*}
and $\mathbf{U}^\top\mathbf{AU}=\mbox{diag}\left< -1,1,3\right>.$ 
}
\section{Back to the quadratic forms}
\frame{\frametitle{Classificiation with eigenvalues}
\begin{theorem}
    Consider the quadratic form $Q$, and let $B$ denote the corresponding symmetric matrix, i.e.
$Q(x) = ⟨x, Bx⟩$
for every $x ∈ \mathbb{R}^N$
n. Examine the eigenvalues of $B$.
    \begin{itemize}
    \item
     If all eigenvalues are positive, then $Q$ is positive denite.
     \item
     If all eigenvalues are nonnegative and at least one of them is zero, then $Q$
    is positive semidenite.
     \item
     If all eigenvalues are negative, then $Q$ is negative denite.
     \item
     If all eigenvalues are nonpositive and at least one of them is zero, then $Q$
    is negative semidenite.
     \item
     If there are both positive and negative eigenvalues, then $Q$ is indenite.
     \end{itemize}
\end{theorem}
}
\frame{\frametitle{Classification with leading principal minors}
    Let $B$ denote an $n \times n$ real matrix. 
    The determinant of the top left $k \times k$ matrices ($k=1,\cdots,n$) are called \alert{leading principal minors}.
    \begin{theorem}[Sylvester]
    Consider the quadratic form $Q$, and let $B$ denote the corresponding symmetric matrix, i.e.
    $Q(x) = ⟨x, Bx⟩$
    for every $x ∈ \mathbb{R}^N$.

    The quadratic form is positive definite if and only if all of the leading principal minors of $B$ is positive.
    \end{theorem}
}
\frame{\frametitle{Proof}
    Assume first, that $Q$ is positive definite.
    If $\lambda$ is an eigenvalue, and $x$ is an eigenvector such that $Bx=\lambda x$,
    then 
    \[
    0<Q\left( x \right)
    =⟨x, Bx⟩
    =⟨x, \lambda x⟩
    =\lambda ⟨x, x⟩
    =\lambda \|x\|^2
    \]
    which means that all of the eigenvales of $B$ is positive. 

    Realize, that the determinant of a symmetric matrix is the product of all of the eigenvalues.
    We obtained that the determinant of any positive definite matrix is positive.

    Obviously, the top left $k\times k$ submatrix of a positive definite matrix is also positive definite.
    Thus we proved, that all of the principal minors of $B$ is a positive number.
}
\frame{\frametitle{Proof}
Vice versa, assume now that the sequence of the principal minors $\Delta_1,\cdots,\Delta_n$ are all positive numbers.
At this case all of the top left $k\times k$ submatrix is regular, thus all of the columns of any $k \times k$ submatrix
form a linear independent system.
This fact implicates, that during the dyadic decomposition process all of the top left pivot terms are nonzero numbers.

If $\alpha_1,\cdots,\alpha_n$ is the sequence of the pivot terms, then the dyadic decompositon is
\[
    B=\sum_{k=1}^n\alpha_kd_k\cdot d_k^T
\]
We know that the derminant can be computed as the product of the pivot terms (at this case)
thus $\Delta_1=\alpha_1$, and $\Delta_k=\alpha_k\Delta_{k-1}$ if $k>1$.
This fact implicates that all of the pivot terms $\alpha_k$ are positive numbers, thus the quadratic form is positive definite,
which was to be demonstrated.
}
\section{Problems}
\begin{frame}
    \frametitle{Problem 1}
    Diagonalize the matrix
    \[
        A=
        \begin{pmatrix}
            0&1&1\\1&0&1\\1&1&0
        \end{pmatrix}
    \]
\end{frame}

\begin{frame}
    \frametitle{Problem 2}
    Diagonalize the matrix
    \[
        A=
        \begin{pmatrix}
            4&-3&-3\\3&-2&-3\\-1&1&2
        \end{pmatrix}
    \]
    Compute the matrix $A^{10}$.
\end{frame}
\begin{frame}
    \frametitle{Problem 3}
    Diagonalize the matrix
    \[
        A=
        \begin{pmatrix}
            1&1\\0&1
        \end{pmatrix}
    \]
\end{frame}
\begin{frame}
    \frametitle{Problem 4}
    Consider the matrix
    \[
        A=
        \begin{pmatrix}
            1&2\\4&3
        \end{pmatrix}
    \]

    Compute $A^{10}$. 

    Write $A^{10}$ as a linear combination of dyads.
\end{frame}
\begin{frame}
    \frametitle{Problem 5}
    Diagonalize the matrix
    \[
        A=
        \begin{pmatrix}
            0&1&0\\-1&0&0\\0&0&2
        \end{pmatrix}
    \]
\end{frame}

\begin{frame}
    \frametitle{Problem 6}
    Diagonalize the matrix
    \[
        A=
        \begin{pmatrix}
            2&-1&-1\\-1&2&-1\\-1&-1&2
        \end{pmatrix}
    \]
\end{frame}

\begin{frame}{Trace of a matrix}
    \begin{definition}
        If $A$ is a square matrix, then the sum of the diagonal terms of $A$ is called the \emph{trace}
        of matrix $A$.
        Notation: $\tr{A}$.
    \end{definition}

    Prove that the trace of similar matrices are the same. 
    That is for any regular matrix $P$
    \[
        \tr{(P^{-1}AP)}=\tr{A}.
    \]
    Hint: Let us prove first, that $\tr\left( AB \right)=\tr(BA)$ for any square matrices $A$ and $B$.
\end{frame}

\begin{frame}
    \frametitle{Problem 7}
    Diagonalize the matrix
    \[
        A=
        \begin{pmatrix}
            1&1&1\\1&1&1\\1&1&1
        \end{pmatrix}
    \]
\end{frame}

\begin{frame}
    \frametitle{Problem 8}
    Diagonalize the matrix
    \[
        A=
        \begin{pmatrix}
            0&0&0&0\\1&1&1&1\\0&0&0&0\\1&1&1&1
        \end{pmatrix}
    \]
\end{frame}

\begin{frame}{Upper triangular matrices}
    Prove that an upper triangular matrix is regular if and only if the diagonal terms are non zero numbers.

    Prove that $\lambda$ is an eigenvalue of an upper triangular matrix if and only if $\lambda$ is one of the terms of the diagonal of the matrix.
\end{frame}
\end{document}
